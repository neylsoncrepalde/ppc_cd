\documentclass[a4paper, 12pt, openright, oneside, german, french, english, brazil]{abntex2}
\usepackage[brazil]{babel}
\usepackage{graphicx}
\usepackage[utf8]{inputenc}
\usepackage{helvet}
\renewcommand{\familydefault}{\sfdefault}
\usepackage{wrapfig}
\usepackage{lscape}
\usepackage{rotating}
\usepackage{epstopdf}
\usepackage[alf,abnt-and-type=&]{abntex2cite}
\usepackage[a4paper, left=2cm, right=2cm, top=2cm, bottom=2cm]{geometry}
\usepackage{indentfirst}
\usepackage{longtable}
\pagestyle{plain}

\titulo{Projeto Pedagógico do Curso Superior de Tecnologia em Ciência de Dados}
\autor{Centro Universitário Metodista Izabela Hendrix}
\data{Abril, 2018}
\instituicao{Centro Universitário Metodista Izabela Hendrix}
\local{Belo Horizonte}

\begin{document}
\pretextual
\imprimircapa
%\listoffigures
%\listoftables
\newpage
\tableofcontents

\textual

\part{Contextualização}

\chapter{Dados de Identificação}

\section{Da Instituição Mantenedora}
\textbf{Nome:} Instituto Metodista Izabela Hendrix

\textbf{CNPJ:} 17.217.191/0001-40

\textbf{Endereço:} Rua da Bahia, nº. 2.020, Bairro Funcionários, Belo Horizonte – Minas Gerais.

\textbf{CEP:} 30160-012

\textbf{Telefone:} (31) 3244-7200

\textbf{E-mail:} reitoria@izabelahendrix.metodista.br

\textbf{Home page:} \url{http://www.izabelahendrix.edu.br}

\section{Da Instituição Mantida}
\textbf{Nome:} Centro Universitário Metodista Izabela Hendrix

\textbf{Endereço:} Campus Praça da Liberdade: Rua da Bahia, nº. 2.020, Bairro Funcionários, Belo Horizonte – Minas Gerais.

\textbf{CEP:} 30160-012

\textbf{Telefone:} (31) 3244-7200

\textbf{E-mail:} reitoria@izabelahendrix.metodista.br

\textbf{Home page:} \url{http://www.izabelahendrix.edu.br}

\section{Dos Dirigentes}

\textbf{Diretor Geral:} Robson Ramos de Aguiar

\textbf{Reitor:} Luciano Sathler Rosa Guimarães

\textbf{Diretoria Acadêmica:} Cecília Maria Carvalho Soares Oliveira

\textbf{Gestor Administrativo:} Marcos Antônio dos Reis


\chapter{Histórico}

\section{Histórico Institucional}

O Centro Universitário Metodista Izabela Hendrix faz parte de uma rede com mais de 700 universidades e colleges em todos os continentes, adota este estilo: valoriza sua memória, mas é original e projeta o futuro a cada instante.

O movimento metodista surgiu como renovação da Igreja Anglicana, na Inglaterra, na primeira metade dos anos 1700, durante a Revolução Industrial, dentro da Universidade de Oxford. Dali expandiu-se para outros continentes tendo ao lado de suas Igrejas uma escola e organizações de promoção humana\footnote{Ainda hoje existem, nas Igrejas Metodistas, as escolas dominicais voltadas para as crianças de até 12 anos.}. Revela, desde as suas origens, a concepção teológico-filosófica de unidade entre a prática da fé e a formação educativa para a vida social.

No Brasil, o processo de constituição de um sistema próprio de educação metodista tem origem no processo de inserção do protestantismo histórico no país; no contexto da abertura dos portos e substituição da mão-de-obra escrava pela mão-de-obra remunerada artesanal, manufatureira e laboral da imigração européia. O trabalho educacional era estratégia para o estabelecimento da Igreja no país, convicta de que a educação metodista ajudaria a libertação da ignorância, do analfabetismo e promoveria a modernização da sociedade, objetivando preparar as novas gerações para virem a ser futuras lideranças nacionais. As primeiras escolas metodistas se localizaram nos focos de presença da Igreja, nos Estados de São Paulo, Rio de Janeiro, Minas Gerais e Rio Grande do Sul.

O Izabela Hendrix foi criado em 1904, pela missionária norte-americana Marta Watts. No início, era um colégio destinado à educação de mulheres, o primeiro de Minas Gerais a reconhecer seus direitos e compreender a importância da atuação feminina na sociedade. Com mais de um século de existência, o Izabela Hendrix tem uma história que se mistura com a trajetória de Belo Horizonte. A capital mineira, com 110 anos apresenta em seu traçado e na sua gente a sofisticação e a ousadia de quem busca o novo sem deixar de ter raízes.

Hoje, o Instituto Metodista Izabela Hendrix é Colégio, Centro Universitário (Metodista de Minas), possui três campi e segue acreditando no poder transformador do ensino de qualidade. Com estrutura moderna e inteligente, professores(as) de alto nível e uma atuação arrojada no cenário educacional, orientada pelos princípios de uma sociedade mais justa e igualitária, a Metodista de Minas caminha para se tornar Universidade.

Mas estrutura e diretrizes não representam nada se não produzirem idéias e soluções inovadoras. Para isso, mais de 6.000 pessoas dão vida à Metodista de Minas. São estudantes, professores(as) e demais funcionários(as) que, junto à comunidade, usam laboratórios, bibliotecas, clínicas, auditórios, parque esportivo, capela, salas de aula e espaços de convívio como instrumentos que formam cidadãos e cidadãs e trazem melhorias para a sociedade.

Na tradição metodista, as escolas não são apenas lugar de estudar, sendo também espaços para viver. Além dos ambientes já citados, o Izabela Hendrix tem um teatro com 400 lugares, equipado com estrutura para a realização de eventos culturais e acadêmicos. Desde 2007, o Instituto conta com uma ampla biblioteca 24 horas, mais um projeto pioneiro que reafirma os ideais desta Casa, onde todo tempo é tempo de aprender e viver, onde a tradição é a base forte e a inovação é o caminho a percorrer.

\section{Histórico do Curso}


\chapter{Marco Referencial}

\section{Marco Referencial Institucional}

\subsection{A instituição}

O Centro Universitário Metodista Izabela Hendrix é uma instituição confessional, comunitária, privada. Sua natureza confessional reside em sua vinculação à Igreja Metodista, que entende a educação como ``o processo que visa a oferecer à pessoa e à comunidade uma compreensão da vida e da sociedade, comprometida com uma prática libertadora, recriando a vida e a sociedade, segundo o modelo de Jesus Cristo, questionando os sistemas de dominação e morte, à luz do Reino de Deus''\footnote{Diretrizes para a Educação na Igreja Metodista, Cânones da Igreja Metodista, 2002.}. A atuação educacional da Igreja Metodista não tem interesse pecuniário, nem desejo de proselitismo. Antes, é uma fidelidade para com sua vocação histórica e missionária.

Por sua vez, entendida a Igreja Metodista como uma comunidade missionária a serviço do povo, a natureza comunitária do Centro Universitário Metodista Izabela Hendrix origina-se de sua confessionalidade. Sua ação educativa, portanto, buscará sempre a melhoria das condições de vida no mundo e um posicionamento contra quaisquer tipos de preconceitos e ações de discriminação e exclusão.

Finalmente, sua natureza privada decorre do fato de ter sido instituída por uma entidade não governamental, dispondo, para sua manutenção e desenvolvimento, majoritariamente de recursos próprios.

\subsection{Missão}

A missão do Centro Universitário Metodista Izabela Hendrix é educar e formar cidadãos qualificados e críticos, com base em valores cristãos, para atuar na transformação da sociedade. Educar e formar revelam o desejo de transcender o processo educativo, que é muito centrado na pessoa, e buscar também o processo formativo, ou seja, a construção dos futuros cidadãos, conscientes de seus direitos e deveres e dotados de sólidos valores morais e éticos.

Por sua vez, o objetivo de formar cidadãos qualificados incorpora a dimensão da competência técnica, pois, parte importante da formação em nível superior é o “aprender a fazer”, isto é, a aquisição de competências específicas que vão definir o capacidade de pensar com autonomia e independência, exercendo juízo com acuidade.

A base dos valores cristãos está no cerne do Izabela Hendrix, como instituição confessional da Igreja Metodista, fundada por John Wesley - criador também da primeira escola metodista, a Kingswood School (1748 – Inglaterra) – que afirmava que ``ou teria uma escola cristã ou não teria nenhuma''. Ressalte-se que ser uma escola baseada em valores cristãos não significa dedicar-se ao proselitismo nem se mostrar intolerante para com pessoas de outras denominações e confissões religiosas. Pelo contrário, o princípio wesleyano era o ``pensar e deixar pensar''. Afirmar valores cristãos significa defender a justiça, a solidariedade, a cidadania, aspectos que o Izabela Hendrix considera imprescindíveis para que as pessoas sejam completas em sua formação.

Finalmente, a missão contextualiza o âmbito e o propósito da ação dos egressos da Instituição como para atuar na transformação da sociedade. A educação, no Izabela Hendrix, será sempre direcionada para gerar nos aprendentes o sentimento de inconformismo e o desejo por mudanças. O aluno que tem acesso à educação superior de qualidade precisa ter consciência de sua responsabilidade e compromisso para com os que não têm condições de ter o mesmo benefício.

\subsection{Práticas Educacionais}

A educação é uma jornada espiritual ou o processo de construção da verdade. Seu propósito essencial, a partir da aquisição do conhecimento, é religar o ser humano àquilo que de outro modo seria difícil ou inacessível, para inseri-lo, novamente, na grande trama da existência.

John Wesley, fundador do Metodismo, afirmava que “o propósito do conhecimento é permitir que os homens alcem seus pensamentos a objetos cada vez mais elevados e dignos de consideração, até ascenderem à fonte de todo o conhecimento, Deus”. Assim, em termos de suas práticas educacionais, o Centro

Universitário Metodista Izabela Hendrix busca os seguintes objetivos e características:

\begin{itemize}
\item Ser uma comunidade aprendente, elegendo o aluno como protagonista principal do processo educativo e não concebendo o professor como simples emissor de informa-ções, arcaísmo indesejado diante de um mundo saturado. Repudia-se o mero recurso à memorização como simulação da inteligência, entendendo que a inteligência é uma função que só se ativa na presença de uma situação-problema, exigindo flexibilidade e pensamento criativo.
\item Entender o desejo de conhecer como insaciável, ``um princípio fundamental no ser humano, inserido em sua natureza mais íntima.'' (John Wesley). 
\item Estimular o pensamento crítico como modo de participação do cidadão e a tolerância como meio de ouvir os outros sem perder a própria voz.
\item Promover a indissociabilidade entre ensino, pesquisa e extensão, ação que deve ter seu início na sala de aula, formando-se o aluno-pesquisador. Este deve ter na atividade de indagação o desafio para a descoberta de soluções novas.
\item Refletir, permanentemente, sobre a responsabilidade social do profissional formado em nível superior.
\item Conceber a interdisciplinaridade como forma de despertar o interesse e o compromisso dos alunos com o conhecimento, evitando-se a alienação causada pela frag-mentação dos conteúdos.
\item Incorporar em todas as suas práticas acadêmicas uma boa variedade de técnicas e recursos didáticos, sempre em busca do engajamento do aluno no processo de ensino-aprendizagem.
\item Incentivar reflexões sobre o papel das novas tecnologias na sociedade e no próprio processo de ensino-aprendizagem.
\item Conceber as práticas avaliativas como objeto fundamental para o desenvolvimento intelectual e pessoal do aprendente.
\end{itemize}

\subsection{Linhas Curriculares Institucionais - LCI's}

Uma visão mais consistente do endereçamento das políticas acadêmicas desencadeou o processo de definição das Linhas Curriculares Institucionais (LCIs). Para tal, o currículo foi entendido em sua visão macro, conforme o consagrado nessa área de conhecimento: isto é, tudo o que se produz no Centro Universitário Metodista Izabela Hendrix com a intencionalidade educativa. Assim, as LCIs nortearão o percurso de todos os cursos em suas atividades de ensino, pesquisa e extensão. Serão as grandes ênfases curriculares institucionais que refletirão, em síntese, qual é a intenção e missão da Educação Superior na Metodista de Minas.

As linhas, dentro das vocações da casa, devem expressar ao mesmo tempo os reflexos da história (o instituído) e o presente (instituído-instituinte), sinalizando o rumo que se quer imprimir às ações (instituinte de um ``novo'' instituído).

As LCIs não se confundem com a linhas de pesquisa ou áreas de conhecimento. Por serem mais amplas, as LCIs abarcam as linhas de pesquisa e as áreas de conhecimento e estende suas ações às linhas do ensino e da extensão. As Linhas Curriculares Institucionais são:

\begin{itemize}
\item Meio ambiente
\item Espaço Urbano
\item Responsabilidade social e formação cidadã
\item Gestão e Liderança
\item Empreendedorismo e inovação
\item Saúde coletiva
\end{itemize}

Dentro dos conteúdos da matriz curricular do curso de Ciência de Dados, serão trabalhadas as LCIs conforme descrito abaixo:

\begin{itemize}
\item Meio ambiente: entender a análise de dados ambientais dentro de uma visão sustentável;
\item Espaço urbano: compreender o espaço urbano dentro do contexto interno e externo à organização e suas implicações na gestão e tomada de decisão incorporando o conceito de \textit{cidades inteligentes};
\item Responsabilidade social e formação cidadã: formar e desenvolver um profissional com princípios de responsabilidade social, ética e cidadã;
\item Gestão e liderança: desenvolver competências sólidas para gerir e liderar pessoas e processos de maneira profissional com base em resultados;
\item Saúde coletiva: entender a importância da prevenção e saúde física e mental do trabalhador e sua produtividade na empresa;
\item Empreendedorismo e inovação: desenvolver o espírito empreendedor e o intraempreendedorismo, bem como capacitar o aluno para elaboração de um plano de negócio e tomar decisão.
\end{itemize}

As LCIs são eixos articuladores do conhecimento que permitem a convergência de práticas de ensino, pesquisa e extensão entre cursos, corpo docente e discente da institui-ção.

\section{Marco Referencial do Curso}

\subsection{O Curso Superior de Tecnologia em Ciência de Dados - Perfil do Profissional}

O Curso Superior de Tecnologia em Ciência de Dados é o primeiro curso de Graduação do Brasil; obedece as diretrizes contidas no parecer da Conselho Nacional de Educação (CNE)/Câmara de Educação Superior (CES) 436/2001; apresenta caráter formativo multidisciplinar, ultrapassando as áreas das ciências exatas; possui ferramentas que levarão o aluno a identificar, compilar, analisar e comunicar dados complexos para nas mais diversas áreas contribuindo para a melhoria da qualidade de vida e inovações no país.

Durante o curso, o aluno terá a certificação de sua capacitação durante os módulos (Art.5 CNE/CP 3/2002); obtenção de diploma de tecnólogo no término do curso; orientação prática em todas as disciplinas; e será capaz de induzir e participar do desenvolvimento em análise e processamento de dados do país, visando suprir a crescente demanda por profissionais capacitados em tecnologias alternativas para atuar na solução de problemas e criação de ferramentas básicas e aplicadas.

O desenvolvimento das capacidades profissionais do egresso será baseado na integra-ção da Educação, Ciência e Tecnologia, garantido a formação de cidadãos com competências profissionais inovadoras e éticas em prol do desenvolvimento do País.

As competências do egresso serão baseadas de acordo com a LDB (Lei 9394/96), o Decreto 5.591/2005 e o Art. 2 da CNE/CP3.

Durante o curso o aluno irá:

\begin{itemize}
\item Compreender os processos tecnológicos em suas causas e efeitos e desenvolver sua capacidade empreendedora;
\item Entender a importância da produção, inovação, gestão e incorporação científico-tecno-lógica para o mercado de trabalho, proporcionando a compreensão e avaliação dos impactos sociais, econômicos e ambientais resultantes;
\item Desenvolver competências profissionais tecnológicas, gerais e específicas, para a gestão de processos e a produção de bens e serviços;
\item Conscientizar sobre a importância do estudo contínuo (ex.: Pós-Graduação), aprimoramento das capacidades profissionais;
\item Entender a importância da interdisciplinaridade e atualização constante dos cursos e currículos;
\item Compreender as mais recentes ferramentas e tópicos de ciências de dados incluindo-se arquiteturas de \textit{Big Data}, captura e sistematização de dados, \textit{Data Mining}, Inteligência Artificial, os principais algoritmos de \textit{Machine Learning} e \textit{Deep Learning} e técnicas de visualização de dados;
\item Compreender a importância da ciência de dados na área das inovações tecnológicas, como no desenvolvimento de novos softwares e produtos;
\item Desenvolver competências com linguagens de programação como Python, R, entre outros;
\item Aprimorar uma carreira em ciências de dados, pesquisa, indústria, educação e ou desenvolvimento de software;
\end{itemize}

Sob o ponto de vista pedagógico, este projeto adotará as seguintes diretrizes:

\begin{itemize}
\item O currículo e o projeto pedagógico serão avaliados continuamente, visando subsidiar as correções necessárias;
\item A formação pretendida em Ciências de Dados tem caráter generalista e, para tanto, o currículo apresenta projetos integradores em todos os módulos, proporcionando visão interdisciplinar do Curso;
\item O currículo estimulará, em consonância com as particularidades das disciplinas, a realização de projetos de pesquisa, projetos de extensão e de atividades práticas que possibilitem ao aluno o acesso às diversas formas do conhecimento científico e sua aplicação na melhoria da qualidade de vida da comunidade. Serão também estimuladas as atividades que socializem o conhecimento produzido pelos corpos docente e discente;
\item Os conteúdos programáticos relativos às disciplinas e todas as atividades decorrentes destes estarão automática e permanentemente submetidas a uma análise crítica visando à identificação dos elementos essenciais (amplos) e acessórios (específicos); os primeiros terão o seu domínio garantido para cada área do conhecimento ou atua-ção;
\item Cada disciplina apresentará claramente, em seu plano de ensino, os seus objetivos gerais e específicos, os conteúdos, a metodologia de ensino e estratégias de avaliação;
\item Os estágios livres (não obrigatórios) nas diversas áreas de atuação do Cientista de Dados serão estimulados.
\end{itemize}

\subsection{Missão}

A missão proposta pelo Curso Tecnológico em Ciências de Dados do Centro Universitário Metodista Izabela Hendrix é a formação de profissionais cuja competência ultrapasse o âmbito das suas especialidades, sendo capazes de desenvolver projetos de pesquisa, produtos, novas metodologias de forma crítica e inovadora nas áreas de Inteligência Artificial, Big Data e Inteligência Organizacional. Suprindo a crescente demanda por profissionais capacitados em tecnologias alternativas para atuar na solução de problemas e criação de ferramentas básicas e aplicadas nas mais diversas áreas, contribuindo para a melhoria da qualidade de vida.

\subsection{Contexto atual da Ciência de Dados}

O Cientista de Dados é um profissional multidisciplinar que aplica técnicas da computa-ção, estatística e inteligência artificial no sentido de análise de dados para gerar informação e conhecimento nas mais diversas áreas.

O avanço da capacidade de armazenamento e processamento de dados das últimas duas décadas aumentou o número e a complexidade de dados gerados. O Cientista de Dados passou a exercer papel fundamental na criação, desenvolvimento e operação de banco de dados e outras ferramentas computacionais para coletar, organizar e interpretar dados.

Atualmente este profissional é capaz de trabalhar com aquisição de dados, processamento, armazenamento, distribuição, análise, interpretação e programação, com o objetivo de entender padrões e elaborar previsões com grande acurácia.

A atuação do Cientista de Dados no mercado de trabalho ou na carreira acadêmica tem acompanhado o crescimento da geração de dados e do processamento de informatização de empresas e centros voltados às áreas da indústria, agropecuária, inovação e pesquisas. A demanda por profissionais nesta área é grande, sendo considerada a grande profissão do futuro.

O profissional poderá trabalhar na área industrial, financeira, em centros de pesquisa, centros médicos, clínicas, órgãos públicos e empresas privadas.

\subsection{Perspectivas}

Este projeto pretende viabilizar condições (Art.7 CNE/CP 3/2002) para que os alunos do Curso Superior de Tecnologia em Ciência de Dados do Centro Universitário Metodista Izabela Hendrix possam efetivamente adquirir e colocar em prática a competência profissional pretendida de forma eficiente e eficaz requeridas para o desenvolvimento tecnológico.

Promover a formação de profissionais que visem os avanços tecnológicos, o aperfeiçoa-mento e as inovações para gerar ferramentas e produtos que contribuam para o desenvolvimento do país.


\chapter{Contextualização}

O curso Superior de Tecnologia em Ciência de Dados do Centro Universitário Metodista Izabela Hendrix pretende a formação integral de um profissional. Para tanto, precisa articular os saberes produzidos aos saberes necessários para a formação de profissionais. Estes saberes transcendem os relativos ao conteúdo das disciplinas. É fundamental que este sujeito seja despertado para a importância do seu papel na manutenção ou transformação da ordem social vigente.

Desta forma, é muito importante buscar a unidade dos saberes. Tratá-los de forma dual (científicos x senso comum, teóricos x práticos, racional x emocional), acaba por reforçar a dissociação, que fragiliza a formação.

É parte fundamental da contexto de aplicação deste projeto garantir o que preconiza a Lei de Diretrizes e Bases para a Educação Nacional quando define como fins da educação superior:

\begin{citacao}
  (\dots) estimular a criação cultural e o desenvolvimento do espírito científico e do pensamento reflexivo; formar diplomados nas diferentes áreas de conhecimento, aptos para a inserção em setores profissionais e para a participação no desenvolvimento da sociedade brasileira, e colaborar na sua formação contínua; incentivar o trabalho de pesquisa e investigação científica, visando o desenvolvimento da ciência e da tecnologia e da criação e difusão da cultura, e, desse modo, desenvolver o entendimento do homem e do meio em que vive; promover a divulgação de conhecimentos culturais, científicos e técnicos que constituem patrimônio da humanidade e comunicar o saber através do ensino, de publicações ou de outras formas de comunicação; suscitar o desejo permanente de aperfeiçoamento cultural e profissional e possibilitar a correspondente concretização, integrando os conhecimentos que vão sendo adquiridos numa estrutura intelectual sistematizadora do conhecimento de cada geração; estimular o conhecimento dos problemas do mundo presente, em particular os nacionais e regionais, prestar serviços especializados à comunidade e estabelecer com esta uma relação de reciprocidade; promover a extensão, aberta à participação da população, visando à difusão das conquistas e benefícios resultantes da criação cultural e da pesquisa científica e tecnológica geradas na instituição.\footnote{Ministério da Educação e Cultura, 1996.}
\end{citacao}

Acreditamos que a função social da formação profissional está voltada também para responder as demandas da sociedade, sejam elas advindas das indústrias ou do setor de serviços. É importante que seja superada a idéia de que as teorias sejam possíveis só no contexto acadêmico, mas, e principalmente, que a ciência esteja a serviço do desenvolvimento social.

Este projeto pedagógico pretende garantir que a graduação em Ciência de Dados seja uma etapa inicial, a base do processo de educação continuada. Para tanto, propomos que a formação integral do sujeito profissional requer sustentação por uma diversidade de espaços complementares de formação, sejam estes representados pelas atividades complementares, sejam constituídos pela convivência interdisciplinar.

O Curso Superior de Tecnologia em Ciência de Dados do Centro Universitário Metodista Izabela Hendrix foi idealizado tendo o paradigma da complexidade como norteador epistemológico. Isso reflete-se na lógica de organização da matriz curricular, que evita a linearidade, situação em que o conhecimento resultaria de uma sequência de disciplinas com conteúdos que dependiam necessariamente de uma disciplina anterior. Nossa proposta opta o quanto possível por uma lógica não linear, que pode ser compreendida sob dois aspectos.

No primeiro deles, todas as disciplinas do curso são divididas entre disciplinas de formação básica e disciplinas de formação específica. Dentro da formação básica, desdobram-se três dimensões. A primeira delas abarca um conjunto de disciplinas que trata da Relação ser humano-sociedade. A segunda formada pelo conjunto de disciplinas que irão promover a formação específica. Finalmente, a terceira dimensão da formação técnico-instrumental, em que o conjunto de disciplinas que tratam da produção do conhecimento científico e tecnológico.

Tal dimensão é fundamental para o desenvolvimento do pensamento científico e por métodos investigativos que serão a base das relações com a busca pelo conhecimento e a informação.

Na formação específica encontram-se outras três dimensões: a primeira delas, a dimensão Informática. Tais conhecimento irão municiar os alunos de conhecimentos para levar à sociedade uma atuação comprometida com as boas práticas de cada modalidade. Temos também a dimensão Técnico-instrumental irá abarcar disciplinas que darão à esse aluno ferramentas didáticas para o trabalho com os conhecimentos específicos da área, levando em conta também os princípios de inclusão e acessibilidade aos seus alunos. Finalmente, a dimensão prático-pedagógica contempla a disciplinas que contribuem para a mediação do conhecimento e a atuação profissional, tal como os estágios e práticas como componente curricular, responsáveis por aproximar os alunos da realidade de atuação na sociedade.

Uma formação que não pretende ser linear requer que mesmo os espaços tradicionais do processo de formação sejam repensados. A sala de aula deve ser pensada como a menor unidade deste processo; ela deve ser reconhecida como o endereço, como o local de referência para um sujeito em formação. Este espaço deve ser ocupado por um conjunto de iguais; um local de onde todos irão partir; um local onde se reencontrarão em diferentes etapas da caminhada; um local onde a diversidade será reconhecida como potencialidade e as especificidades como complementares. A pesquisa, por sua vez, deverá orientar, instigar, despertar o espírito explorador da natureza humana; deve dar a direção e os instrumentos para a busca de conhecimentos e a produção do novo. Ela deve ser adotada como uma dimensão do processo de formação, um elemento imprescindível à formação que se pretende.

A articulação dos conhecimentos pode ser explicitada na prática de pesquisa. Para além de aprenderem a metodologia de pesquisa é fundamental que vivenciem o processo da pesquisa desde a exploração do tema até o produto final. Esta ideia de que a pesquisa é uma prática para poucos ou que existem os que pensam, pesquisam e produzem conhecimentos e os que executam, deve ser superada, pois pode ser mais um elemento de reforço da dissociação do conhecimento e da fragilidade da formação profissional.

Os laboratórios e espaços específicos do curso precisam ser locais de convivência e de troca interdisciplinar, onde as diversas áreas do saber possam cooperar, onde aprendam a conviver de forma harmônica e solidária como cidadãos e onde aprendam a compartilhar as descobertas e comungar os sucessos e insucessos, onde reconheçam que as respostas e o conhecimento ora produzido são provisórios, onde a verdade não existe como absoluta, onde a teoria se materializa, onde enfim possamos nos reconhecer como iguais.

\chapter{Perspectivas}

Com a implantação do Curso Superior de Tecnologia em Ciência de Dados o Centro Universitário Metodista Izabela Hendrix alia a tradição metodista com sua missão social, se comprometendo com a formação de profissionais que atenderão às necessidades da sociedade de forma crítica e criativa.

A implantação deste curso amplia a oferta de cursos, mostra um reposicionamento da imagem da instituição, e incentiva o desenvolvimento de mecanismos de promoção e difusão cultural e artística articulando as comunidades Izabelina e a local.

A localização do Campus Praça da Liberdade, no qual estão situados também os cursos de Bacharelado em Engenharia da Computação, Engenharia de Produção, Engenharia Civil, Engenharia Ambiental e Sanitária, Arquitetura e os cursos Tecnológicos de Bioinformática e Design de interiores, favorece o acesso aos trabalhadores e estudantes de toda a região metropolitana, onde cumpre importante papel social.


\part{Do Curso}

\chapter{Dados sobre o curso}

\section{Regulamentação Legal do Profissional}

\section{Integralização}

\section{Número de Vagas}

\section{Turno de Funcionamento}



\chapter{Objetivos do Curso}

\chapter{Perfil do Profissional Formado pelo Centro Universitário Metodista Izabela Hendrix}

\section{Competências e Habilidades Comuns}

\section{Competências e Habilidades Específicas do Curso}


\chapter{Práticas de Ensino}

\chapter{Práticas de Pesquisa}

\chapter{Práticas de Extensão}

\chapter{Práticas de Internacionalização}



\part{Das Práticas Pedagógicas}

\chapter{Matriz Curricular}


\section{Distribuição das disciplinas por dimensões de conhecimento}


\section{Plano de Curso}

\section{Atividades Práticas Supervisionadas (APS's)}



\chapter{interdisciplinaridade}

\section{Políticas e Práticas de Educação à Distância}


\chapter{Infraestrutura}


\chapter{Avaliação Docente e Discente}


\chapter{Ementário}



\end{document}
